% Required packages
\usepackage{tcolorbox}
\usepackage{hyperref}



% Avoid expanding it when in a textual label
\MakeRobust{\ref}

% Enables labeling text and directly linking to it. Courtesy of https://tex.stackexchange.com/a/292890
\makeatletter
\newcommand{\labeltext}[2]{%
  \@bsphack
  \csname phantomsection\endcsname % in case hyperref is used
  \def\@currentlabel{#1}{\label{#2}}%
  \@esphack
}
\makeatother

% Highlighting box around a specified text (argument 1). The fallback colour for the box is magenta, but another
% colour can be specified (argument 2) if needed.
%   arg1           : To-be highlighted text
%   arg2 (optional): Box colour
%           example: \highlight[cyan]{This is some text}
\newtcbox{\highlight}[1][magenta]{on line, arc=5pt,colback=#1!10!white,colframe=#1!50!black, before upper={\rule[-3pt]{0pt}{10pt}},boxrule=1pt, boxsep=0pt,left=4pt,right=3pt,top=2pt,bottom=1pt}

% ATTENTION: CASE SENSITIVE!
% Creates a research question for the paper. This entails a highlighted box with a label for the question and an 
% italic text that is "saved" for later referencing. An invisible version of the text is placed at the position of
% the original text with a big label around it for easy referencing and linking to.
%     arg1: RQ-Label
%     arg2: Text of the research question
%  example: \researchq{RQ1}{This is a \textbf{research} question!}
\newcommand{\researchq}[2]{
    \noindent % No indent please :)
    \highlight{\textbf{\texttt{\textcolor{gray}{#1}}}} % Creates the coloured box for the label
    \textit{#2} % Displays the actual text of the question in italics, keeping all additional formatting
    \labeltext{\highlight{\textbf{\texttt{\textcolor{gray}{#1}}}} \textit{#2}} % "Saving" the rq incl. question text. It's placed invisibly at the position of the actual question, so it can be easily referenced and linked to.
    {rq:#1} % The specific label for this rq; label-name is equivalent to "rq:" + arg1
}

% ATTENTION: CASE SENSITIVE!
% Pretty wrapper function to reference a research question. 
%     arg1: The rq that is to be referenced
%  example: \refrq{RQ1}
\newcommand{\refrq}[1]{\ref{rq:#1}}

% ATTENTION: CASE SENSITIVE!
% Show the label of the research question without a label-reference link and without the text.
%     arg1: The rq that is to be referenced
%  example: \norefrq{RQ1}
\newcommand{\norefrq}[1]{\highlight{\textbf{\texttt{\textcolor{gray}{#1}}}}}

% ATTENTION: CASE SENSITIVE!
% Show the label of the research question with a label-reference link but without the text.
%     arg1: The rq that is to be referenced
%  example: \notextrefrq{RQ1}
\newcommand{\notextrefrq}[1]{\hyperref[rq:#1]{\highlight{\textbf{\texttt{\textcolor{gray}{#1}}}}}}
